\documentclass{scrartcl}

\usepackage{ucs}
\usepackage[utf8x]{inputenc}
\usepackage[ngerman]{babel}
\usepackage[hidelinks]{hyperref}
\usepackage{graphicx}
\usepackage{amsmath}
\usepackage{amssymb}
\usepackage{color}
\usepackage{listings}

\lstset{frame=tb,
language=R,
keywordstyle=\color{blue},
alsoletter={.}
}

\setlength\parindent{0pt}

\title{Grundlagen der Statistik \\ Zusammenfassung}
\author{Thomas Mohr}
\date{}

\begin{document}
\maketitle
\pagebreak
\tableofcontents{}
\pagebreak

\section{Univariate deskriptive Statstik}

\subsection{Grundbegriffe}

\begin{itemize}
	\item Merkmalsträger (Beobachtungseinheiten) \\
	Objekte einer Stichprobe, z.B. Personen, Geräte, etc.
	\item Merkmale (Variablen) \\
	Interessierende Größe, Eigenschaften der Beobachtungseinheiten, z.B. Alter, Geschlecht, Vorliegen eines Defektes, etc.
	\item Merkmalsausprägungen \\
	Die verschiedenen Werte, die ein Merkmal annehmen kann, z.B. männlich/weiblich, Defekt ja/nein, etc.
\end{itemize}

\subsection{Merkmalstypen}

\begin{tabular}{|c|c|}
\hline 
dichotom & Nur zwei Ausprägungen \\ 
\hline 
nominal & Kategorien \textbf{ohne} Reihenfolge \\ 
\hline 
ordinal & Kategorien \textbf{mit} Reihenfolge \\ 
\hline 
diskret & Endlich oder abzählbar unendlich viele Ausprägungen \\ 
\hline 
stetig & Alle Werte in einem reelen Interval als Ausprägung \\ 
\hline 
\end{tabular} 

\subsection{Merkmalstypen in R}

\begin{tabular}{|c|c|}
\hline 
\textbf{Merkmalstyp} & \textbf{Variablentyp in R} \\ 
\hline 
nominal & factor \\ 
\hline 
ordinal & ordered factor \\ 
\hline 
diskret & integer \\ 
\hline 
stetig & double \\ 
\hline 
\end{tabular} 

\subsection{Aufgaben}



\end{document}