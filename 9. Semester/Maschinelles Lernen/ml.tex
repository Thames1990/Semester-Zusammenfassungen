\documentclass{scrartcl}

\usepackage{ucs}
\usepackage[utf8x]{inputenc}
\usepackage[english]{babel}
\usepackage{graphicx}
\usepackage{amsmath}
\usepackage{amssymb}
\usepackage{hyperref}

\setlength\parindent{0pt}

\title{Maschinelles Lernen \\ Zusammenfassung}
\author{Thomas Mohr}
\date{}

\begin{document}
\maketitle
\pagebreak
\tableofcontents
\pagebreak

\section{Grundlagen}

\subsection{(Un)-überwachtes Lernen}

\begin{itemize}
	\item Eine \textbf{überwachte} Lernaufgabe liegt vor, wenn wir Beispiele 
	haben, die das zu lernende Attribut bereits tragen (Zielvariable).
	\begin{itemize}
		\item \textbf{Regression} im Fall von kontinuierlichen Werten (z.B. $ 
		\mathbb{R} $)
		\item \textbf{Klassifikation} im Fall von diskreten Labeln (z.B. 
		\textit{TRUE, FALSE}; ausgezeichnet, durchschnittlich, schlecht)
	\end{itemize}
	\item Eine \textbf{unüberwachte} Lernaufgabe liegt vor, wenn es kein 
	Attribut gibt, das wir lernen wollen und für das wir bereits Beispiele 
	haben.
	\begin{itemize}
		\item Clustering, also die Unterteilung der Daten in eine Menge 
		von Gruppen
		\item Finden von Ausreißern
	\end{itemize}
\end{itemize}

\subsection{Inkrementelles Lernen}

\begin{itemize}
	\item Anstatt das Modell stets von Null an zu lernen, wird das alte Modell 
	mit neuen Beispielen erweitert.
	\item 
\end{itemize}

\subsection{Aktives Lernen}

\begin{itemize}
	\item Aktive Lernverfahren erzeugen die Beispiele selbst, d.h., sie sagen 
	dem Benutzer, welches Tupel benötigt wird.
\end{itemize}

\subsection{Data cleansing}

\begin{itemize}
	\item Fehlende Werte auffüllen
	\item Rauschen aus den Daten entfernen
	\item Daten glätten
	\item Ausreißer entfernen
	\item Identische Tupel identifizieren
	\item Daten komprimieren
\end{itemize}

\subsection{Datensatz}

\begin{itemize}
	\item Ein Datensatz ist eine Tabelle
	\item Eine Instanz (auch Objekt) ist eine Zeile in dieser Tabelle
	\item Ein Attribut ist ein Feld, das ein Merkmal des Objekts repräsentiert. 
	Mögliche Arten von Attributen sind
	\begin{itemize}
		\item nominal (kategorisch)
		\begin{itemize}
			\item Keine sinnvolle Ordnung
			\item Wir können nicht rechnen (z.B. Mittelwert, Median, Abstände)
		\end{itemize}
		\item ordinal (sortierte Kategorien)
		\begin{itemize}
			\item Sinnvolle Ordnung
			\item Der Unterschied zwischen zwei Ausprägungen ist i.d.R. 
			unbekannt
		\end{itemize}
		\item binär
		\begin{itemize}
			\item Können nur zwei Werte annehmen
		\end{itemize}
		\item numerisch
		\begin{itemize}
			\item Messbare Quantitäten
			\item Abstand zwischen zwei Werten kann quantifiziert werden
			\item Auf den Attributen kann gerechnet werden
			\item Wir unterscheiden
			\begin{itemize}
				\item diskrete Attribute (endliche oder abzählbar unendliche 
				Menge von womöglichen Ausprägungen)
				\item kontinuierliche Werte, reele Zahlen
				\item Attribute mit echtem Nullpunkt (Gewicht, Größe)
				\item Attribute ohne echten Nullpunkt (Jahresangaben, 
				Temperatur in °C)
			\end{itemize}
		\end{itemize}
	\end{itemize}
	\item Ein Datensatz besitzt $ N $ Instanzen und $ d $ Attribute
	\begin{itemize}
		\item $ x_i $ beschreibt die $ i $-te Instanz
		\item $ x_{ij} $ beschreibt das $ j $-te Attribut der $ i $-ten Instanz
		\item $ x $ beschreibt einen $ d $-dimensionalen Vektor
		\item Liegt eine überwachte Lernaufgabe vor, so ist das Label der $ i 
		$-ten Instanz $ t_i $
	\end{itemize}
\end{itemize}

\section{Deskriptive Statistik}

\section{Regression}

\section{Klassifikation}

\section{Clustering}

\section{Warenkorbanalyse}

\section{Analyse von Graphdaten}

\end{document}